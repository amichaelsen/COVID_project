\documentclass[11pt]{article}

\usepackage{sectsty}
\usepackage{graphicx}

% Margins
\topmargin=-0.45in
\evensidemargin=0in
\oddsidemargin=0in
\textwidth=6.5in
\textheight=9.0in
\headsep=0.25in

\title{DATA 200 Graduate Project\\ Topic 1: Dataset A}
\author{ Anya Michaelsen (3034964414) }
\date{}

\begin{document}
\maketitle	
%\pagebreak

% Optional TOC
 \tableofcontents
% \pagebreak

%--Paper--

%\section{Report Requirements}
%
%
%
%The narrative notebook should include figures sparingly to support specific claims. It can include runnable components, but it should not have large amounts of code. The length of the report should be 8+/-2 pages when it is printed as a PDF, excluding figures and code.
%
%

\pagebreak
\section{Background}


\subsection{The COVID-19 Pandemic}

COVID-19 is airborne respiratory disease caused by SARS-CoV-2, which originated in China and has spread to a global pandemic. Symptoms range from mild or even asymptomatic to fatal. While mortality rates for COVID-19 are still being estimated, scientists believe COVID-19 to be substantially more deadly than most strains of flu.% https://www.hopkinsmedicine.org/health/conditions-and-diseases/coronavirus/coronavirus-disease-2019-vs-the-flu
 As an airborne illness, COVID-19 spreads through droplets in the air, making it highly contagious, and asymptomatic infection combined with up to two week incubation time before symptoms arise make slowing the spread of the virus a public health challenge. 
 
 In December of 2019, the first cases of COVID-19 were detected in Wuhan, China. About twenty days later, the Center for Disease Control (CDC) confirmed the first case in the United States, with the first death following about a month after that. 
 
 Initially, there was no vaccine for the novel coronavirus, and public health measures included mask wearing and social distancing from others to prevent transmission. At a national level, travel bans were implemented to reduce transmission between countries, in particular slowing the spread from countries with high case rates. 
 
 In the United States, measures such as social distancing and masking were quickly politicized, slowing their adoption and mitigating their effectiveness. Mid-March of 2020, the US declared a state of national emergency and some states, such as California, issues Stay-at-Home orders which required people to stay home unless necessary. Over the next year states implemented a variety of measures to stop the spread, including similar stay-at-home orders, mask mandates in public spaces. 
 
 In December of 2020, the the first COVID-19 vaccine was approved for emergency use by the Food and Drug Administration (FDA) in the United States. This vaccine, by Pfizer, used a novel vaccination approach that had been developed over years prior to the COVID-19 pandemic. This method required two doses for full vaccination, spaced several weeks apart, and the vaccine itself required special handling that increased distribution challenges. A week later another vaccine by Moderna, applying a similar inoculation strategy was approved for emergency use. A third vaccine, by Johnson \& Johnson was later approved in March that required only a single shot and more typical storage requirements. 
 During vaccine roll-outs, approval and recommendations were often stratified by age, medical conditions, and exposure risks. 
 
	
 
 \subsection{COVID-19 Data Tracking}
	
	Tracking case numbers, hospitalizations, deaths, and symptom severity has been critical for political bodies making public health decisions as well as for scientists and health care professionals treating and combating the virus. Reporting systems vary globally, both in which metrics are tracked and often how they are defined. Within the United States, case numbers were not centrally tracked at a national level and left to states, which created further disparities in data reporting. Journalists, data scientists, and health institutions took up the mantle of aggregating COVID-19 case data until a national framework could be put in place. 
	
	One such database was created and maintained by the  Center for Systems Science and Engineering (CSSE) at Johns Hopkins University and posted publicly on GitHub, which combines state level data for COVID-19 cases from April of 2020 through March of 2021. 
	
	By the time vaccines had been developed and approved for use in the United States, the CDC was prepared and able to track roll out in a centralized manner. 
	

\subsection{Research Questions}
%Clearly stated research questions and why they are interesting and important. You must include at least one research question involving at least one or more datasets from one of the topics we provided, but you may include additional research questions about each individual dataset. At least one of your research questions has to include a modeling component, e.g., can we build a model using climate data to predict growth in COVID-19 cases accurately?

While tracking COVID-19 cases has been crucial for public health policy, it is also important to both predict cases going forward to implement preventative measures, such as mask wearing, social distancing, and increased vaccination, as well as understand the \textit{causes} of transmission to create effective measures and loosen ineffective restrictive ones. These aims would simultaneously save lives, health care costs, and limit unnecessary restrictions on people's lives as much as possible. 

Throughout the course of the pandemic in the United States, there have been several significant spikes in COVID-19 cases. Possible causes include variants of the virus that are either more transmissible, more deadly, or both, increased travel during holiday months, anti-masking and anti-vaccine rhetoric and mentality in some regions/populations, changes in weather affecting social gathering patterns, and more. 

The goal of this research is to produce models for COVID-19 cases, as measured by `Confirmed Cases' using state level data for COVID-19 metrics, and explore the effects of several possible variables, including weather temperature data,  the distribution of cases by age, and adjacent state COVID-19 metrics. 


\subsubsection{Affects of Weather}


\noindent\textbf{Question 1: Can weather data, both current and historical averages, be used to improve state-level models for the spread of COVID-19?}

Create and train models using COVID numbers only, then 

\subsubsection{Affects of Age-Distribution}


\noindent\textbf{Question 2: Does the ratio of COVID-19 cases by age have any significant effect in predicting COVID-19 deaths?}

compute ratios of 65+ cases/total cases for each month (smallest granularity in the dataset :/) and incorporate the ratios from last month into the model and look for improvements 

\subsubsection{Affects of Adjacent States}

\noindent\textbf{Question 3: Does incorporating COVID data from adjacent states significantly improve our state level COVID models? }

compute aggregates for state adjacency numbers and put into the model. Compare model metrics. 

%\noindent\textbf{Question 3: Can we incorporate vaccination data to update our models for early 2021 to improve accuracy?}

%examine the traffic project and ``change-point" methods more for ideas on how to handle this question 

\subsection{Literature Review}
A brief survey of related work on the topic(s) of your analysis and how your project differs from or complements existing research.

\section{Methodology}

Methodology: carefully describe the methods you use and why they are appropriate for answering your search questions. It must include
\begin{itemize}
\item a brief overview of causal inference, which should be written in a way such that another student in Data 100 who has never been exposed to the concept can carry out the analyses involving the datasets in your project.
\item a detailed description of how modeling is done in your project, including inference or prediction methods used, feature engineering and regularization if applicable, and cross-validation or test data as appropriate for model selection and evaluation.
\end{itemize}


\subsection{The Data}

The primary datasets for this analysis pertain to COVID-19 cases in the United States from April 2020 through March 2021.  

\subsubsection*{COVID-19 Cases Data}


\subsubsection*{Weather Data}

\subsubsection*{Vaccination Data}



\subsection{Data Processing}
Lorem Ipsum \\

\subsubsection{Data Cleaning}
Lorem Ipsum \\

\subsubsection{Feature Engineering}
Lorem Ipsum \\


\subsubsection{Causal Inference}
a brief overview of causal inference, which should be written in a way such that another student in Data 100 who has never been exposed to the concept can carry out the analyses involving the datasets in your project.

write up "causal inference" for modeling, i.e. bootstrapping confidence intervals for coefficients and what that means. 


https://data102.org/sp20/assets/notes/notes13.pdf

want to determine effects... 
but can have confounding effects (example with graphs) 
want to control for these confounding variables, but not mediators (intermediate variables creating a pathway of effect from one variable to another). 

Can hold constant confounding variables and look at the effects within this population? 


\subsubsection{Modeling}

ex-ante vs post-ante forecasting article:\\
https://otexts.com/fpp2/forecasting-regression.html\#ex-ante-versus-ex-post-forecasts


\section{Results}

\subsubsection{Dataset Findings}
 Interesting findings* about each dataset when analyzed individually. Include visualizations and descriptions of data cleaning and data transformation necessary to perform the analysis that led to your findings.
 
 Interesting findings* involving your datasets. Include visualizations and descriptions of data cleaning and data transformation necessary to perform the analysis that led to your findings. 
 
 
* Examples of interesting findings: interesting data distributions and trends, correlations between different features, the relationship between the data distribution for the general population and specific datasets (e.g., the gender distribution in the census dataset vs. in the mental health dataset), specific features that are notably effective/ineffective for prediction.
 
\subsubsection{Model Findings}
Analysis of your findings to answer your research question(s). Include visualizations and specific results. If your research questions contain a modeling component, you must compare the results using different inference or prediction methods (e.g., linear regression, logistic regression, or classification and regression trees). Can you explain why some methods performed better than others?


\subsubsection{Limitations and Future Research}
 An evaluation of your approach and discuss any limitations of the methods you used.

Describe any surprising discoveries that you made and future work.


\section{Appendices}
\section{References}


%--/Paper--

\end{document}